\documentclass[11pt]{article}

    \usepackage[breakable]{tcolorbox}
    \usepackage{parskip} % Stop auto-indenting (to mimic markdown behaviour)
    
    \usepackage{iftex}
    \ifPDFTeX
    	\usepackage[T1]{fontenc}
    	\usepackage{mathpazo}
    \else
    	\usepackage{fontspec}
    \fi

    % Basic figure setup, for now with no caption control since it's done
    % automatically by Pandoc (which extracts ![](path) syntax from Markdown).
    \usepackage{graphicx}
    % Maintain compatibility with old templates. Remove in nbconvert 6.0
    \let\Oldincludegraphics\includegraphics
    % Ensure that by default, figures have no caption (until we provide a
    % proper Figure object with a Caption API and a way to capture that
    % in the conversion process - todo).
    \usepackage{caption}
    \DeclareCaptionFormat{nocaption}{}
    \captionsetup{format=nocaption,aboveskip=0pt,belowskip=0pt}

    \usepackage{float}
    \floatplacement{figure}{H} % forces figures to be placed at the correct location
    \usepackage{xcolor} % Allow colors to be defined
    \usepackage{enumerate} % Needed for markdown enumerations to work
    \usepackage{geometry} % Used to adjust the document margins
    \usepackage{amsmath} % Equations
    \usepackage{amssymb} % Equations
    \usepackage{textcomp} % defines textquotesingle
    % Hack from http://tex.stackexchange.com/a/47451/13684:
    \AtBeginDocument{%
        \def\PYZsq{\textquotesingle}% Upright quotes in Pygmentized code
    }
    \usepackage{upquote} % Upright quotes for verbatim code
    \usepackage{eurosym} % defines \euro
    \usepackage[mathletters]{ucs} % Extended unicode (utf-8) support
    \usepackage{fancyvrb} % verbatim replacement that allows latex
    \usepackage{grffile} % extends the file name processing of package graphics 
                         % to support a larger range
    \makeatletter % fix for old versions of grffile with XeLaTeX
    \@ifpackagelater{grffile}{2019/11/01}
    {
      % Do nothing on new versions
    }
    {
      \def\Gread@@xetex#1{%
        \IfFileExists{"\Gin@base".bb}%
        {\Gread@eps{\Gin@base.bb}}%
        {\Gread@@xetex@aux#1}%
      }
    }
    \makeatother
    \usepackage[Export]{adjustbox} % Used to constrain images to a maximum size
    \adjustboxset{max size={0.9\linewidth}{0.9\paperheight}}

    % The hyperref package gives us a pdf with properly built
    % internal navigation ('pdf bookmarks' for the table of contents,
    % internal cross-reference links, web links for URLs, etc.)
    \usepackage{hyperref}
    % The default LaTeX title has an obnoxious amount of whitespace. By default,
    % titling removes some of it. It also provides customization options.
    \usepackage{titling}
    \usepackage{longtable} % longtable support required by pandoc >1.10
    \usepackage{booktabs}  % table support for pandoc > 1.12.2
    \usepackage[inline]{enumitem} % IRkernel/repr support (it uses the enumerate* environment)
    \usepackage[normalem]{ulem} % ulem is needed to support strikethroughs (\sout)
                                % normalem makes italics be italics, not underlines
    \usepackage{mathrsfs}
    

    
    % Colors for the hyperref package
    \definecolor{urlcolor}{rgb}{0,.145,.698}
    \definecolor{linkcolor}{rgb}{.71,0.21,0.01}
    \definecolor{citecolor}{rgb}{.12,.54,.11}

    % ANSI colors
    \definecolor{ansi-black}{HTML}{3E424D}
    \definecolor{ansi-black-intense}{HTML}{282C36}
    \definecolor{ansi-red}{HTML}{E75C58}
    \definecolor{ansi-red-intense}{HTML}{B22B31}
    \definecolor{ansi-green}{HTML}{00A250}
    \definecolor{ansi-green-intense}{HTML}{007427}
    \definecolor{ansi-yellow}{HTML}{DDB62B}
    \definecolor{ansi-yellow-intense}{HTML}{B27D12}
    \definecolor{ansi-blue}{HTML}{208FFB}
    \definecolor{ansi-blue-intense}{HTML}{0065CA}
    \definecolor{ansi-magenta}{HTML}{D160C4}
    \definecolor{ansi-magenta-intense}{HTML}{A03196}
    \definecolor{ansi-cyan}{HTML}{60C6C8}
    \definecolor{ansi-cyan-intense}{HTML}{258F8F}
    \definecolor{ansi-white}{HTML}{C5C1B4}
    \definecolor{ansi-white-intense}{HTML}{A1A6B2}
    \definecolor{ansi-default-inverse-fg}{HTML}{FFFFFF}
    \definecolor{ansi-default-inverse-bg}{HTML}{000000}

    % common color for the border for error outputs.
    \definecolor{outerrorbackground}{HTML}{FFDFDF}

    % commands and environments needed by pandoc snippets
    % extracted from the output of `pandoc -s`
    \providecommand{\tightlist}{%
      \setlength{\itemsep}{0pt}\setlength{\parskip}{0pt}}
    \DefineVerbatimEnvironment{Highlighting}{Verbatim}{commandchars=\\\{\}}
    % Add ',fontsize=\small' for more characters per line
    \newenvironment{Shaded}{}{}
    \newcommand{\KeywordTok}[1]{\textcolor[rgb]{0.00,0.44,0.13}{\textbf{{#1}}}}
    \newcommand{\DataTypeTok}[1]{\textcolor[rgb]{0.56,0.13,0.00}{{#1}}}
    \newcommand{\DecValTok}[1]{\textcolor[rgb]{0.25,0.63,0.44}{{#1}}}
    \newcommand{\BaseNTok}[1]{\textcolor[rgb]{0.25,0.63,0.44}{{#1}}}
    \newcommand{\FloatTok}[1]{\textcolor[rgb]{0.25,0.63,0.44}{{#1}}}
    \newcommand{\CharTok}[1]{\textcolor[rgb]{0.25,0.44,0.63}{{#1}}}
    \newcommand{\StringTok}[1]{\textcolor[rgb]{0.25,0.44,0.63}{{#1}}}
    \newcommand{\CommentTok}[1]{\textcolor[rgb]{0.38,0.63,0.69}{\textit{{#1}}}}
    \newcommand{\OtherTok}[1]{\textcolor[rgb]{0.00,0.44,0.13}{{#1}}}
    \newcommand{\AlertTok}[1]{\textcolor[rgb]{1.00,0.00,0.00}{\textbf{{#1}}}}
    \newcommand{\FunctionTok}[1]{\textcolor[rgb]{0.02,0.16,0.49}{{#1}}}
    \newcommand{\RegionMarkerTok}[1]{{#1}}
    \newcommand{\ErrorTok}[1]{\textcolor[rgb]{1.00,0.00,0.00}{\textbf{{#1}}}}
    \newcommand{\NormalTok}[1]{{#1}}
    
    % Additional commands for more recent versions of Pandoc
    \newcommand{\ConstantTok}[1]{\textcolor[rgb]{0.53,0.00,0.00}{{#1}}}
    \newcommand{\SpecialCharTok}[1]{\textcolor[rgb]{0.25,0.44,0.63}{{#1}}}
    \newcommand{\VerbatimStringTok}[1]{\textcolor[rgb]{0.25,0.44,0.63}{{#1}}}
    \newcommand{\SpecialStringTok}[1]{\textcolor[rgb]{0.73,0.40,0.53}{{#1}}}
    \newcommand{\ImportTok}[1]{{#1}}
    \newcommand{\DocumentationTok}[1]{\textcolor[rgb]{0.73,0.13,0.13}{\textit{{#1}}}}
    \newcommand{\AnnotationTok}[1]{\textcolor[rgb]{0.38,0.63,0.69}{\textbf{\textit{{#1}}}}}
    \newcommand{\CommentVarTok}[1]{\textcolor[rgb]{0.38,0.63,0.69}{\textbf{\textit{{#1}}}}}
    \newcommand{\VariableTok}[1]{\textcolor[rgb]{0.10,0.09,0.49}{{#1}}}
    \newcommand{\ControlFlowTok}[1]{\textcolor[rgb]{0.00,0.44,0.13}{\textbf{{#1}}}}
    \newcommand{\OperatorTok}[1]{\textcolor[rgb]{0.40,0.40,0.40}{{#1}}}
    \newcommand{\BuiltInTok}[1]{{#1}}
    \newcommand{\ExtensionTok}[1]{{#1}}
    \newcommand{\PreprocessorTok}[1]{\textcolor[rgb]{0.74,0.48,0.00}{{#1}}}
    \newcommand{\AttributeTok}[1]{\textcolor[rgb]{0.49,0.56,0.16}{{#1}}}
    \newcommand{\InformationTok}[1]{\textcolor[rgb]{0.38,0.63,0.69}{\textbf{\textit{{#1}}}}}
    \newcommand{\WarningTok}[1]{\textcolor[rgb]{0.38,0.63,0.69}{\textbf{\textit{{#1}}}}}
    
    
    % Define a nice break command that doesn't care if a line doesn't already
    % exist.
    \def\br{\hspace*{\fill} \\* }
    % Math Jax compatibility definitions
    \def\gt{>}
    \def\lt{<}
    \let\Oldtex\TeX
    \let\Oldlatex\LaTeX
    \renewcommand{\TeX}{\textrm{\Oldtex}}
    \renewcommand{\LaTeX}{\textrm{\Oldlatex}}
    % Document parameters
    % Document title
    \title{ice\_thickness\_eda}
    
    
    
    
    
% Pygments definitions
\makeatletter
\def\PY@reset{\let\PY@it=\relax \let\PY@bf=\relax%
    \let\PY@ul=\relax \let\PY@tc=\relax%
    \let\PY@bc=\relax \let\PY@ff=\relax}
\def\PY@tok#1{\csname PY@tok@#1\endcsname}
\def\PY@toks#1+{\ifx\relax#1\empty\else%
    \PY@tok{#1}\expandafter\PY@toks\fi}
\def\PY@do#1{\PY@bc{\PY@tc{\PY@ul{%
    \PY@it{\PY@bf{\PY@ff{#1}}}}}}}
\def\PY#1#2{\PY@reset\PY@toks#1+\relax+\PY@do{#2}}

\expandafter\def\csname PY@tok@w\endcsname{\def\PY@tc##1{\textcolor[rgb]{0.73,0.73,0.73}{##1}}}
\expandafter\def\csname PY@tok@c\endcsname{\let\PY@it=\textit\def\PY@tc##1{\textcolor[rgb]{0.25,0.50,0.50}{##1}}}
\expandafter\def\csname PY@tok@cp\endcsname{\def\PY@tc##1{\textcolor[rgb]{0.74,0.48,0.00}{##1}}}
\expandafter\def\csname PY@tok@k\endcsname{\let\PY@bf=\textbf\def\PY@tc##1{\textcolor[rgb]{0.00,0.50,0.00}{##1}}}
\expandafter\def\csname PY@tok@kp\endcsname{\def\PY@tc##1{\textcolor[rgb]{0.00,0.50,0.00}{##1}}}
\expandafter\def\csname PY@tok@kt\endcsname{\def\PY@tc##1{\textcolor[rgb]{0.69,0.00,0.25}{##1}}}
\expandafter\def\csname PY@tok@o\endcsname{\def\PY@tc##1{\textcolor[rgb]{0.40,0.40,0.40}{##1}}}
\expandafter\def\csname PY@tok@ow\endcsname{\let\PY@bf=\textbf\def\PY@tc##1{\textcolor[rgb]{0.67,0.13,1.00}{##1}}}
\expandafter\def\csname PY@tok@nb\endcsname{\def\PY@tc##1{\textcolor[rgb]{0.00,0.50,0.00}{##1}}}
\expandafter\def\csname PY@tok@nf\endcsname{\def\PY@tc##1{\textcolor[rgb]{0.00,0.00,1.00}{##1}}}
\expandafter\def\csname PY@tok@nc\endcsname{\let\PY@bf=\textbf\def\PY@tc##1{\textcolor[rgb]{0.00,0.00,1.00}{##1}}}
\expandafter\def\csname PY@tok@nn\endcsname{\let\PY@bf=\textbf\def\PY@tc##1{\textcolor[rgb]{0.00,0.00,1.00}{##1}}}
\expandafter\def\csname PY@tok@ne\endcsname{\let\PY@bf=\textbf\def\PY@tc##1{\textcolor[rgb]{0.82,0.25,0.23}{##1}}}
\expandafter\def\csname PY@tok@nv\endcsname{\def\PY@tc##1{\textcolor[rgb]{0.10,0.09,0.49}{##1}}}
\expandafter\def\csname PY@tok@no\endcsname{\def\PY@tc##1{\textcolor[rgb]{0.53,0.00,0.00}{##1}}}
\expandafter\def\csname PY@tok@nl\endcsname{\def\PY@tc##1{\textcolor[rgb]{0.63,0.63,0.00}{##1}}}
\expandafter\def\csname PY@tok@ni\endcsname{\let\PY@bf=\textbf\def\PY@tc##1{\textcolor[rgb]{0.60,0.60,0.60}{##1}}}
\expandafter\def\csname PY@tok@na\endcsname{\def\PY@tc##1{\textcolor[rgb]{0.49,0.56,0.16}{##1}}}
\expandafter\def\csname PY@tok@nt\endcsname{\let\PY@bf=\textbf\def\PY@tc##1{\textcolor[rgb]{0.00,0.50,0.00}{##1}}}
\expandafter\def\csname PY@tok@nd\endcsname{\def\PY@tc##1{\textcolor[rgb]{0.67,0.13,1.00}{##1}}}
\expandafter\def\csname PY@tok@s\endcsname{\def\PY@tc##1{\textcolor[rgb]{0.73,0.13,0.13}{##1}}}
\expandafter\def\csname PY@tok@sd\endcsname{\let\PY@it=\textit\def\PY@tc##1{\textcolor[rgb]{0.73,0.13,0.13}{##1}}}
\expandafter\def\csname PY@tok@si\endcsname{\let\PY@bf=\textbf\def\PY@tc##1{\textcolor[rgb]{0.73,0.40,0.53}{##1}}}
\expandafter\def\csname PY@tok@se\endcsname{\let\PY@bf=\textbf\def\PY@tc##1{\textcolor[rgb]{0.73,0.40,0.13}{##1}}}
\expandafter\def\csname PY@tok@sr\endcsname{\def\PY@tc##1{\textcolor[rgb]{0.73,0.40,0.53}{##1}}}
\expandafter\def\csname PY@tok@ss\endcsname{\def\PY@tc##1{\textcolor[rgb]{0.10,0.09,0.49}{##1}}}
\expandafter\def\csname PY@tok@sx\endcsname{\def\PY@tc##1{\textcolor[rgb]{0.00,0.50,0.00}{##1}}}
\expandafter\def\csname PY@tok@m\endcsname{\def\PY@tc##1{\textcolor[rgb]{0.40,0.40,0.40}{##1}}}
\expandafter\def\csname PY@tok@gh\endcsname{\let\PY@bf=\textbf\def\PY@tc##1{\textcolor[rgb]{0.00,0.00,0.50}{##1}}}
\expandafter\def\csname PY@tok@gu\endcsname{\let\PY@bf=\textbf\def\PY@tc##1{\textcolor[rgb]{0.50,0.00,0.50}{##1}}}
\expandafter\def\csname PY@tok@gd\endcsname{\def\PY@tc##1{\textcolor[rgb]{0.63,0.00,0.00}{##1}}}
\expandafter\def\csname PY@tok@gi\endcsname{\def\PY@tc##1{\textcolor[rgb]{0.00,0.63,0.00}{##1}}}
\expandafter\def\csname PY@tok@gr\endcsname{\def\PY@tc##1{\textcolor[rgb]{1.00,0.00,0.00}{##1}}}
\expandafter\def\csname PY@tok@ge\endcsname{\let\PY@it=\textit}
\expandafter\def\csname PY@tok@gs\endcsname{\let\PY@bf=\textbf}
\expandafter\def\csname PY@tok@gp\endcsname{\let\PY@bf=\textbf\def\PY@tc##1{\textcolor[rgb]{0.00,0.00,0.50}{##1}}}
\expandafter\def\csname PY@tok@go\endcsname{\def\PY@tc##1{\textcolor[rgb]{0.53,0.53,0.53}{##1}}}
\expandafter\def\csname PY@tok@gt\endcsname{\def\PY@tc##1{\textcolor[rgb]{0.00,0.27,0.87}{##1}}}
\expandafter\def\csname PY@tok@err\endcsname{\def\PY@bc##1{\setlength{\fboxsep}{0pt}\fcolorbox[rgb]{1.00,0.00,0.00}{1,1,1}{\strut ##1}}}
\expandafter\def\csname PY@tok@kc\endcsname{\let\PY@bf=\textbf\def\PY@tc##1{\textcolor[rgb]{0.00,0.50,0.00}{##1}}}
\expandafter\def\csname PY@tok@kd\endcsname{\let\PY@bf=\textbf\def\PY@tc##1{\textcolor[rgb]{0.00,0.50,0.00}{##1}}}
\expandafter\def\csname PY@tok@kn\endcsname{\let\PY@bf=\textbf\def\PY@tc##1{\textcolor[rgb]{0.00,0.50,0.00}{##1}}}
\expandafter\def\csname PY@tok@kr\endcsname{\let\PY@bf=\textbf\def\PY@tc##1{\textcolor[rgb]{0.00,0.50,0.00}{##1}}}
\expandafter\def\csname PY@tok@bp\endcsname{\def\PY@tc##1{\textcolor[rgb]{0.00,0.50,0.00}{##1}}}
\expandafter\def\csname PY@tok@fm\endcsname{\def\PY@tc##1{\textcolor[rgb]{0.00,0.00,1.00}{##1}}}
\expandafter\def\csname PY@tok@vc\endcsname{\def\PY@tc##1{\textcolor[rgb]{0.10,0.09,0.49}{##1}}}
\expandafter\def\csname PY@tok@vg\endcsname{\def\PY@tc##1{\textcolor[rgb]{0.10,0.09,0.49}{##1}}}
\expandafter\def\csname PY@tok@vi\endcsname{\def\PY@tc##1{\textcolor[rgb]{0.10,0.09,0.49}{##1}}}
\expandafter\def\csname PY@tok@vm\endcsname{\def\PY@tc##1{\textcolor[rgb]{0.10,0.09,0.49}{##1}}}
\expandafter\def\csname PY@tok@sa\endcsname{\def\PY@tc##1{\textcolor[rgb]{0.73,0.13,0.13}{##1}}}
\expandafter\def\csname PY@tok@sb\endcsname{\def\PY@tc##1{\textcolor[rgb]{0.73,0.13,0.13}{##1}}}
\expandafter\def\csname PY@tok@sc\endcsname{\def\PY@tc##1{\textcolor[rgb]{0.73,0.13,0.13}{##1}}}
\expandafter\def\csname PY@tok@dl\endcsname{\def\PY@tc##1{\textcolor[rgb]{0.73,0.13,0.13}{##1}}}
\expandafter\def\csname PY@tok@s2\endcsname{\def\PY@tc##1{\textcolor[rgb]{0.73,0.13,0.13}{##1}}}
\expandafter\def\csname PY@tok@sh\endcsname{\def\PY@tc##1{\textcolor[rgb]{0.73,0.13,0.13}{##1}}}
\expandafter\def\csname PY@tok@s1\endcsname{\def\PY@tc##1{\textcolor[rgb]{0.73,0.13,0.13}{##1}}}
\expandafter\def\csname PY@tok@mb\endcsname{\def\PY@tc##1{\textcolor[rgb]{0.40,0.40,0.40}{##1}}}
\expandafter\def\csname PY@tok@mf\endcsname{\def\PY@tc##1{\textcolor[rgb]{0.40,0.40,0.40}{##1}}}
\expandafter\def\csname PY@tok@mh\endcsname{\def\PY@tc##1{\textcolor[rgb]{0.40,0.40,0.40}{##1}}}
\expandafter\def\csname PY@tok@mi\endcsname{\def\PY@tc##1{\textcolor[rgb]{0.40,0.40,0.40}{##1}}}
\expandafter\def\csname PY@tok@il\endcsname{\def\PY@tc##1{\textcolor[rgb]{0.40,0.40,0.40}{##1}}}
\expandafter\def\csname PY@tok@mo\endcsname{\def\PY@tc##1{\textcolor[rgb]{0.40,0.40,0.40}{##1}}}
\expandafter\def\csname PY@tok@ch\endcsname{\let\PY@it=\textit\def\PY@tc##1{\textcolor[rgb]{0.25,0.50,0.50}{##1}}}
\expandafter\def\csname PY@tok@cm\endcsname{\let\PY@it=\textit\def\PY@tc##1{\textcolor[rgb]{0.25,0.50,0.50}{##1}}}
\expandafter\def\csname PY@tok@cpf\endcsname{\let\PY@it=\textit\def\PY@tc##1{\textcolor[rgb]{0.25,0.50,0.50}{##1}}}
\expandafter\def\csname PY@tok@c1\endcsname{\let\PY@it=\textit\def\PY@tc##1{\textcolor[rgb]{0.25,0.50,0.50}{##1}}}
\expandafter\def\csname PY@tok@cs\endcsname{\let\PY@it=\textit\def\PY@tc##1{\textcolor[rgb]{0.25,0.50,0.50}{##1}}}

\def\PYZbs{\char`\\}
\def\PYZus{\char`\_}
\def\PYZob{\char`\{}
\def\PYZcb{\char`\}}
\def\PYZca{\char`\^}
\def\PYZam{\char`\&}
\def\PYZlt{\char`\<}
\def\PYZgt{\char`\>}
\def\PYZsh{\char`\#}
\def\PYZpc{\char`\%}
\def\PYZdl{\char`\$}
\def\PYZhy{\char`\-}
\def\PYZsq{\char`\'}
\def\PYZdq{\char`\"}
\def\PYZti{\char`\~}
% for compatibility with earlier versions
\def\PYZat{@}
\def\PYZlb{[}
\def\PYZrb{]}
\makeatother


    % For linebreaks inside Verbatim environment from package fancyvrb. 
    \makeatletter
        \newbox\Wrappedcontinuationbox 
        \newbox\Wrappedvisiblespacebox 
        \newcommand*\Wrappedvisiblespace {\textcolor{red}{\textvisiblespace}} 
        \newcommand*\Wrappedcontinuationsymbol {\textcolor{red}{\llap{\tiny$\m@th\hookrightarrow$}}} 
        \newcommand*\Wrappedcontinuationindent {3ex } 
        \newcommand*\Wrappedafterbreak {\kern\Wrappedcontinuationindent\copy\Wrappedcontinuationbox} 
        % Take advantage of the already applied Pygments mark-up to insert 
        % potential linebreaks for TeX processing. 
        %        {, <, #, %, $, ' and ": go to next line. 
        %        _, }, ^, &, >, - and ~: stay at end of broken line. 
        % Use of \textquotesingle for straight quote. 
        \newcommand*\Wrappedbreaksatspecials {% 
            \def\PYGZus{\discretionary{\char`\_}{\Wrappedafterbreak}{\char`\_}}% 
            \def\PYGZob{\discretionary{}{\Wrappedafterbreak\char`\{}{\char`\{}}% 
            \def\PYGZcb{\discretionary{\char`\}}{\Wrappedafterbreak}{\char`\}}}% 
            \def\PYGZca{\discretionary{\char`\^}{\Wrappedafterbreak}{\char`\^}}% 
            \def\PYGZam{\discretionary{\char`\&}{\Wrappedafterbreak}{\char`\&}}% 
            \def\PYGZlt{\discretionary{}{\Wrappedafterbreak\char`\<}{\char`\<}}% 
            \def\PYGZgt{\discretionary{\char`\>}{\Wrappedafterbreak}{\char`\>}}% 
            \def\PYGZsh{\discretionary{}{\Wrappedafterbreak\char`\#}{\char`\#}}% 
            \def\PYGZpc{\discretionary{}{\Wrappedafterbreak\char`\%}{\char`\%}}% 
            \def\PYGZdl{\discretionary{}{\Wrappedafterbreak\char`\$}{\char`\$}}% 
            \def\PYGZhy{\discretionary{\char`\-}{\Wrappedafterbreak}{\char`\-}}% 
            \def\PYGZsq{\discretionary{}{\Wrappedafterbreak\textquotesingle}{\textquotesingle}}% 
            \def\PYGZdq{\discretionary{}{\Wrappedafterbreak\char`\"}{\char`\"}}% 
            \def\PYGZti{\discretionary{\char`\~}{\Wrappedafterbreak}{\char`\~}}% 
        } 
        % Some characters . , ; ? ! / are not pygmentized. 
        % This macro makes them "active" and they will insert potential linebreaks 
        \newcommand*\Wrappedbreaksatpunct {% 
            \lccode`\~`\.\lowercase{\def~}{\discretionary{\hbox{\char`\.}}{\Wrappedafterbreak}{\hbox{\char`\.}}}% 
            \lccode`\~`\,\lowercase{\def~}{\discretionary{\hbox{\char`\,}}{\Wrappedafterbreak}{\hbox{\char`\,}}}% 
            \lccode`\~`\;\lowercase{\def~}{\discretionary{\hbox{\char`\;}}{\Wrappedafterbreak}{\hbox{\char`\;}}}% 
            \lccode`\~`\:\lowercase{\def~}{\discretionary{\hbox{\char`\:}}{\Wrappedafterbreak}{\hbox{\char`\:}}}% 
            \lccode`\~`\?\lowercase{\def~}{\discretionary{\hbox{\char`\?}}{\Wrappedafterbreak}{\hbox{\char`\?}}}% 
            \lccode`\~`\!\lowercase{\def~}{\discretionary{\hbox{\char`\!}}{\Wrappedafterbreak}{\hbox{\char`\!}}}% 
            \lccode`\~`\/\lowercase{\def~}{\discretionary{\hbox{\char`\/}}{\Wrappedafterbreak}{\hbox{\char`\/}}}% 
            \catcode`\.\active
            \catcode`\,\active 
            \catcode`\;\active
            \catcode`\:\active
            \catcode`\?\active
            \catcode`\!\active
            \catcode`\/\active 
            \lccode`\~`\~ 	
        }
    \makeatother

    \let\OriginalVerbatim=\Verbatim
    \makeatletter
    \renewcommand{\Verbatim}[1][1]{%
        %\parskip\z@skip
        \sbox\Wrappedcontinuationbox {\Wrappedcontinuationsymbol}%
        \sbox\Wrappedvisiblespacebox {\FV@SetupFont\Wrappedvisiblespace}%
        \def\FancyVerbFormatLine ##1{\hsize\linewidth
            \vtop{\raggedright\hyphenpenalty\z@\exhyphenpenalty\z@
                \doublehyphendemerits\z@\finalhyphendemerits\z@
                \strut ##1\strut}%
        }%
        % If the linebreak is at a space, the latter will be displayed as visible
        % space at end of first line, and a continuation symbol starts next line.
        % Stretch/shrink are however usually zero for typewriter font.
        \def\FV@Space {%
            \nobreak\hskip\z@ plus\fontdimen3\font minus\fontdimen4\font
            \discretionary{\copy\Wrappedvisiblespacebox}{\Wrappedafterbreak}
            {\kern\fontdimen2\font}%
        }%
        
        % Allow breaks at special characters using \PYG... macros.
        \Wrappedbreaksatspecials
        % Breaks at punctuation characters . , ; ? ! and / need catcode=\active 	
        \OriginalVerbatim[#1,codes*=\Wrappedbreaksatpunct]%
    }
    \makeatother

    % Exact colors from NB
    \definecolor{incolor}{HTML}{303F9F}
    \definecolor{outcolor}{HTML}{D84315}
    \definecolor{cellborder}{HTML}{CFCFCF}
    \definecolor{cellbackground}{HTML}{F7F7F7}
    
    % prompt
    \makeatletter
    \newcommand{\boxspacing}{\kern\kvtcb@left@rule\kern\kvtcb@boxsep}
    \makeatother
    \newcommand{\prompt}[4]{
        {\ttfamily\llap{{\color{#2}[#3]:\hspace{3pt}#4}}\vspace{-\baselineskip}}
    }
    

    
    % Prevent overflowing lines due to hard-to-break entities
    \sloppy 
    % Setup hyperref package
    \hypersetup{
      breaklinks=true,  % so long urls are correctly broken across lines
      colorlinks=true,
      urlcolor=urlcolor,
      linkcolor=linkcolor,
      citecolor=citecolor,
      }
    % Slightly bigger margins than the latex defaults
    
    \geometry{verbose,tmargin=1in,bmargin=1in,lmargin=1in,rmargin=1in}
    
    

\begin{document}
    
    \maketitle
    
    

    
    \begin{tcolorbox}[breakable, size=fbox, boxrule=1pt, pad at break*=1mm,colback=cellbackground, colframe=cellborder]
\prompt{In}{incolor}{1}{\boxspacing}
\begin{Verbatim}[commandchars=\\\{\}]
\PY{k+kn}{import} \PY{n+nn}{pandas} \PY{k}{as} \PY{n+nn}{pd}
\PY{k+kn}{from} \PY{n+nn}{pandas\PYZus{}profiling} \PY{k+kn}{import} \PY{n}{ProfileReport}

\PY{k+kn}{import} \PY{n+nn}{altair} \PY{k}{as} \PY{n+nn}{alt}
\PY{k+kn}{import} \PY{n+nn}{import\PYZus{}data}
\end{Verbatim}
\end{tcolorbox}

    \begin{tcolorbox}[breakable, size=fbox, boxrule=1pt, pad at break*=1mm,colback=cellbackground, colframe=cellborder]
\prompt{In}{incolor}{2}{\boxspacing}
\begin{Verbatim}[commandchars=\\\{\}]
\PY{n}{import\PYZus{}data}\PY{o}{.}\PY{n}{download\PYZus{}data}\PY{p}{(}\PY{p}{)}
\end{Verbatim}
\end{tcolorbox}

    \begin{Verbatim}[commandchars=\\\{\}]
Data file already exists, overwrite? (y/n):  n
    \end{Verbatim}

    \begin{Verbatim}[commandchars=\\\{\}]
INFO    38   import\_data User declined to overwrite.
    \end{Verbatim}

    \begin{tcolorbox}[breakable, size=fbox, boxrule=1pt, pad at break*=1mm,colback=cellbackground, colframe=cellborder]
\prompt{In}{incolor}{3}{\boxspacing}
\begin{Verbatim}[commandchars=\\\{\}]
\PY{n}{df} \PY{o}{=} \PY{n}{pd}\PY{o}{.}\PY{n}{read\PYZus{}csv}\PY{p}{(}\PY{l+s+s2}{\PYZdq{}}\PY{l+s+s2}{../data/raw/ice\PYZus{}thickness.csv}\PY{l+s+s2}{\PYZdq{}}\PY{p}{)}
\PY{n}{df}\PY{p}{[}\PY{l+s+s1}{\PYZsq{}}\PY{l+s+s1}{Date}\PY{l+s+s1}{\PYZsq{}}\PY{p}{]} \PY{o}{=} \PY{n}{pd}\PY{o}{.}\PY{n}{DatetimeIndex}\PY{p}{(}\PY{n}{df}\PY{p}{[}\PY{l+s+s1}{\PYZsq{}}\PY{l+s+s1}{Date}\PY{l+s+s1}{\PYZsq{}}\PY{p}{]}\PY{p}{)}
\end{Verbatim}
\end{tcolorbox}

    \hypertarget{summary-of-the-data-set}{%
\section{Summary of the data set}\label{summary-of-the-data-set}}

The
\href{https://www.canada.ca/en/environment-climate-change/services/ice-forecasts-observations/latest-conditions/archive-overview/thickness-data.html}{data
set} is from the Canadian Ice Thickness Program. The data has been
collected weekly since 1947. The program was updated in 2002, so we are
only looking at data prior to the update. Ice thickness is measured to
the nearest centimetre using one of two methods; special auger kit or
hot wire ice thickness gauge.

    \hypertarget{data-overview}{%
\subsubsection{Data overview}\label{data-overview}}

Our data set has a range of dates from 1984 - 1996. There are several
different stations at which measurements are taken.

    \begin{tcolorbox}[breakable, size=fbox, boxrule=1pt, pad at break*=1mm,colback=cellbackground, colframe=cellborder]
\prompt{In}{incolor}{4}{\boxspacing}
\begin{Verbatim}[commandchars=\\\{\}]
\PY{n}{df}
\end{Verbatim}
\end{tcolorbox}

            \begin{tcolorbox}[breakable, size=fbox, boxrule=.5pt, pad at break*=1mm, opacityfill=0]
\prompt{Out}{outcolor}{4}{\boxspacing}
\begin{Verbatim}[commandchars=\\\{\}]
      StationID/ID de station   Station Name/Nom de station       Date  \textbackslash{}
0                         Q25  14A (END BECANCOUR DOCK) Q25 1984-01-07
1                         Q25  14A (END BECANCOUR DOCK) Q25 1984-01-16
2                         Q25  14A (END BECANCOUR DOCK) Q25 1984-01-21
3                         Q25  14A (END BECANCOUR DOCK) Q25 1984-01-28
4                         Q25  14A (END BECANCOUR DOCK) Q25 1984-02-04
{\ldots}                       {\ldots}                           {\ldots}        {\ldots}
51186                     YZF               YELLOWKNIFE YZF 1996-03-29
51187                     YZF               YELLOWKNIFE YZF 1996-04-05
51188                     YZF               YELLOWKNIFE YZF 1996-04-12
51189                     YZF               YELLOWKNIFE YZF 1996-04-19
51190                     YZF               YELLOWKNIFE YZF 1996-04-26

       Ice Thickness/Épaisseur de la glace  Snow depth/Profondeur de la neige  \textbackslash{}
0                                     40.0                                1.0
1                                     49.0                               20.0
2                                     42.0                                8.0
3                                     43.0                               20.0
4                                     41.0                               22.0
{\ldots}                                    {\ldots}                                {\ldots}
51186                                140.0                               18.0
51187                                136.0                               24.0
51188                                144.0                               14.0
51189                                143.0                               10.0
51190                                154.0                                4.0

       Measurement Method/Méthode de mesure  \textbackslash{}
0                                       NaN
1                                       NaN
2                                       NaN
3                                       NaN
4                                       NaN
{\ldots}                                     {\ldots}
51186                                   1.0
51187                                   1.0
51188                                   1.0
51189                                   1.0
51190                                   1.0

       Surface Topology/Topographie de la surface  \textbackslash{}
0                                             NaN
1                                             NaN
2                                             NaN
3                                             NaN
4                                             NaN
{\ldots}                                           {\ldots}
51186                                         0.0
51187                                         0.0
51188                                         0.0
51189                                         0.0
51190                                         0.0

       Cracks and Leads/Fissures et chenaux
0                                       NaN
1                                       NaN
2                                       NaN
3                                       NaN
4                                       NaN
{\ldots}                                     {\ldots}
51186                                   0.0
51187                                   0.0
51188                                   0.0
51189                                   0.0
51190                                   0.0

[51191 rows x 8 columns]
\end{Verbatim}
\end{tcolorbox}
        
    \hypertarget{data-value-ranges}{%
\subsubsection{Data value ranges}\label{data-value-ranges}}

We have 5112 ice thickness measurements. The mean ice thickness over all
dates is \textasciitilde93.26. The standard deviation is
\textasciitilde57.63, and the measurements range from 0 - 345.

    \begin{tcolorbox}[breakable, size=fbox, boxrule=1pt, pad at break*=1mm,colback=cellbackground, colframe=cellborder]
\prompt{In}{incolor}{5}{\boxspacing}
\begin{Verbatim}[commandchars=\\\{\}]
\PY{n}{df}\PY{o}{.}\PY{n}{describe}\PY{p}{(}\PY{p}{)}
\end{Verbatim}
\end{tcolorbox}

            \begin{tcolorbox}[breakable, size=fbox, boxrule=.5pt, pad at break*=1mm, opacityfill=0]
\prompt{Out}{outcolor}{5}{\boxspacing}
\begin{Verbatim}[commandchars=\\\{\}]
       Ice Thickness/Épaisseur de la glace  Snow depth/Profondeur de la neige  \textbackslash{}
count                         51125.000000                       48652.000000
mean                             93.257643                          14.493978
std                              57.632578                          13.532427
min                               0.000000                           0.000000
25\%                              46.000000                           4.000000
50\%                              79.000000                          10.000000
75\%                             135.000000                          21.000000
max                             345.000000                         152.000000

       Measurement Method/Méthode de mesure  \textbackslash{}
count                          15604.000000
mean                               0.981287
std                                0.144664
min                                0.000000
25\%                                1.000000
50\%                                1.000000
75\%                                1.000000
max                                3.000000

       Surface Topology/Topographie de la surface  \textbackslash{}
count                                15425.000000
mean                                     0.599481
std                                      1.582073
min                                      0.000000
25\%                                      0.000000
50\%                                      0.000000
75\%                                      0.000000
max                                      9.000000

       Cracks and Leads/Fissures et chenaux
count                          15428.000000
mean                               0.436349
std                                0.669096
min                                0.000000
25\%                                0.000000
50\%                                0.000000
75\%                                1.000000
max                                9.000000
\end{Verbatim}
\end{tcolorbox}
        
    \hypertarget{data-types-and-completeness}{%
\subsubsection{Data types and
completeness}\label{data-types-and-completeness}}

Each row has a \texttt{Date}, \texttt{Station\ ID}, and a
\texttt{Station\ Name}. There are 66 rows that are missing an
\texttt{Ice\ Thickness} measurement.

    \begin{tcolorbox}[breakable, size=fbox, boxrule=1pt, pad at break*=1mm,colback=cellbackground, colframe=cellborder]
\prompt{In}{incolor}{6}{\boxspacing}
\begin{Verbatim}[commandchars=\\\{\}]
\PY{n}{df}\PY{o}{.}\PY{n}{info}\PY{p}{(}\PY{p}{)}
\end{Verbatim}
\end{tcolorbox}

    \begin{Verbatim}[commandchars=\\\{\}]
<class 'pandas.core.frame.DataFrame'>
RangeIndex: 51191 entries, 0 to 51190
Data columns (total 8 columns):
 \#   Column                                      Non-Null Count  Dtype
---  ------                                      --------------  -----
 0   StationID/ID de station                     51191 non-null  object
 1   Station Name/Nom de station                 51191 non-null  object
 2   Date                                        51191 non-null  datetime64[ns]
 3   Ice Thickness/Épaisseur de la glace         51125 non-null  float64
 4   Snow depth/Profondeur de la neige           48652 non-null  float64
 5   Measurement Method/Méthode de mesure        15604 non-null  float64
 6   Surface Topology/Topographie de la surface  15425 non-null  float64
 7   Cracks and Leads/Fissures et chenaux        15428 non-null  float64
dtypes: datetime64[ns](1), float64(5), object(2)
memory usage: 3.1+ MB
    \end{Verbatim}

    \hypertarget{variables-and-interactions}{%
\subsubsection{Variables and
interactions}\label{variables-and-interactions}}

Most of the rows have the same \texttt{Measurement\ Method}, but there
are some that are missing the method or have a different method. We will
need to make sure we are only using rows with the same measurement in
our sample.

    \begin{tcolorbox}[breakable, size=fbox, boxrule=1pt, pad at break*=1mm,colback=cellbackground, colframe=cellborder]
\prompt{In}{incolor}{7}{\boxspacing}
\begin{Verbatim}[commandchars=\\\{\}]
\PY{n}{df}\PY{p}{[}\PY{l+s+s2}{\PYZdq{}}\PY{l+s+s2}{Measurement Method/Méthode de mesure}\PY{l+s+s2}{\PYZdq{}}\PY{p}{]}\PY{o}{.}\PY{n}{value\PYZus{}counts}\PY{p}{(}\PY{p}{)}
\end{Verbatim}
\end{tcolorbox}

            \begin{tcolorbox}[breakable, size=fbox, boxrule=.5pt, pad at break*=1mm, opacityfill=0]
\prompt{Out}{outcolor}{7}{\boxspacing}
\begin{Verbatim}[commandchars=\\\{\}]
1.0    15278
0.0      310
2.0       14
3.0        2
Name: Measurement Method/Méthode de mesure, dtype: int64
\end{Verbatim}
\end{tcolorbox}
        
    \begin{tcolorbox}[breakable, size=fbox, boxrule=1pt, pad at break*=1mm,colback=cellbackground, colframe=cellborder]
\prompt{In}{incolor}{8}{\boxspacing}
\begin{Verbatim}[commandchars=\\\{\}]
\PY{n}{ProfileReport}\PY{p}{(}\PY{n}{df}\PY{p}{)}
\end{Verbatim}
\end{tcolorbox}

    
    \begin{Verbatim}[commandchars=\\\{\}]
HBox(children=(HTML(value='Summarize dataset'), FloatProgress(value=0.0, max=22.0), HTML(value='')))
    \end{Verbatim}

    
    \begin{Verbatim}[commandchars=\\\{\}]

    \end{Verbatim}

    
    \begin{Verbatim}[commandchars=\\\{\}]
HBox(children=(HTML(value='Generate report structure'), FloatProgress(value=0.0, max=1.0), HTML(value='')))
    \end{Verbatim}

    
    \begin{Verbatim}[commandchars=\\\{\}]

    \end{Verbatim}

    
    \begin{Verbatim}[commandchars=\\\{\}]
HBox(children=(HTML(value='Render HTML'), FloatProgress(value=0.0, max=1.0), HTML(value='')))
    \end{Verbatim}

    
    \begin{Verbatim}[commandchars=\\\{\}]

    \end{Verbatim}

    
    \begin{Verbatim}[commandchars=\\\{\}]
<IPython.core.display.HTML object>
    \end{Verbatim}

    
            \begin{tcolorbox}[breakable, size=fbox, boxrule=.5pt, pad at break*=1mm, opacityfill=0]
\prompt{Out}{outcolor}{8}{\boxspacing}
\begin{Verbatim}[commandchars=\\\{\}]

\end{Verbatim}
\end{tcolorbox}
        
    \hypertarget{exploratory-analysis-of-ice-thickness}{%
\section{Exploratory analysis of Ice
Thickness}\label{exploratory-analysis-of-ice-thickness}}

To better understand our data and to determine how to sample it we
explored: 1. Number of ice thickness measurements per date 2. Number of
stations per date 3. Distribution of ice thickness over all time 4.
Distribution of ice thickness for each date of interest 5. General
change in ice thickness over time

We removed records with \texttt{Measurement\ Method} not equal to 1 in
order to make sure the measurement method we are looking at is
consistent. We also removed all records missing an
\texttt{Ice\ Thickness} measurement.

    \begin{tcolorbox}[breakable, size=fbox, boxrule=1pt, pad at break*=1mm,colback=cellbackground, colframe=cellborder]
\prompt{In}{incolor}{9}{\boxspacing}
\begin{Verbatim}[commandchars=\\\{\}]
\PY{n}{df\PYZus{}filtered} \PY{o}{=} \PY{n}{df}\PY{o}{.}\PY{n}{copy}\PY{p}{(}\PY{p}{)}
\PY{n}{df\PYZus{}filtered} \PY{o}{=} \PY{n}{df\PYZus{}filtered}\PY{o}{.}\PY{n}{rename}\PY{p}{(}\PY{n}{columns}\PY{o}{=}\PY{p}{\PYZob{}}
    \PY{l+s+s2}{\PYZdq{}}\PY{l+s+s2}{StationID/ID de station}\PY{l+s+s2}{\PYZdq{}} \PY{p}{:} \PY{l+s+s2}{\PYZdq{}}\PY{l+s+s2}{station\PYZus{}id}\PY{l+s+s2}{\PYZdq{}}\PY{p}{,} 
    \PY{l+s+s2}{\PYZdq{}}\PY{l+s+s2}{Station Name/Nom de station}\PY{l+s+s2}{\PYZdq{}} \PY{p}{:} \PY{l+s+s2}{\PYZdq{}}\PY{l+s+s2}{station\PYZus{}name}\PY{l+s+s2}{\PYZdq{}}\PY{p}{,} 
    \PY{l+s+s2}{\PYZdq{}}\PY{l+s+s2}{Date}\PY{l+s+s2}{\PYZdq{}} \PY{p}{:} \PY{l+s+s2}{\PYZdq{}}\PY{l+s+s2}{date}\PY{l+s+s2}{\PYZdq{}}\PY{p}{,} 
    \PY{l+s+s2}{\PYZdq{}}\PY{l+s+s2}{Ice Thickness/Épaisseur de la glace}\PY{l+s+s2}{\PYZdq{}} \PY{p}{:} \PY{l+s+s2}{\PYZdq{}}\PY{l+s+s2}{ice\PYZus{}thickness}\PY{l+s+s2}{\PYZdq{}}\PY{p}{,}
    \PY{l+s+s2}{\PYZdq{}}\PY{l+s+s2}{Snow depth/Profondeur de la neige}\PY{l+s+s2}{\PYZdq{}} \PY{p}{:} \PY{l+s+s2}{\PYZdq{}}\PY{l+s+s2}{snow\PYZus{}depth}\PY{l+s+s2}{\PYZdq{}}\PY{p}{,}
    \PY{l+s+s2}{\PYZdq{}}\PY{l+s+s2}{Measurement Method/Méthode de mesure}\PY{l+s+s2}{\PYZdq{}} \PY{p}{:} \PY{l+s+s2}{\PYZdq{}}\PY{l+s+s2}{measurement\PYZus{}method}\PY{l+s+s2}{\PYZdq{}}\PY{p}{,}
    \PY{l+s+s2}{\PYZdq{}}\PY{l+s+s2}{Surface Topology/Topographie de la surface}\PY{l+s+s2}{\PYZdq{}} \PY{p}{:} \PY{l+s+s2}{\PYZdq{}}\PY{l+s+s2}{surface\PYZus{}topology}\PY{l+s+s2}{\PYZdq{}}\PY{p}{,}
    \PY{l+s+s2}{\PYZdq{}}\PY{l+s+s2}{Cracks and Leads/Fissures et chenaux}\PY{l+s+s2}{\PYZdq{}} \PY{p}{:} \PY{l+s+s2}{\PYZdq{}}\PY{l+s+s2}{cracks\PYZus{}leads}\PY{l+s+s2}{\PYZdq{}}
\PY{p}{\PYZcb{}}\PY{p}{)}

\PY{n}{df\PYZus{}filtered} \PY{o}{=} \PY{n}{df\PYZus{}filtered}\PY{p}{[}\PY{n}{df\PYZus{}filtered}\PY{p}{[}\PY{l+s+s2}{\PYZdq{}}\PY{l+s+s2}{measurement\PYZus{}method}\PY{l+s+s2}{\PYZdq{}}\PY{p}{]} \PY{o}{==} \PY{l+m+mi}{1}\PY{p}{]}
\PY{n}{df\PYZus{}filtered} \PY{o}{=} \PY{n}{df\PYZus{}filtered}\PY{p}{[}\PY{n}{df\PYZus{}filtered}\PY{p}{[}\PY{l+s+s2}{\PYZdq{}}\PY{l+s+s2}{ice\PYZus{}thickness}\PY{l+s+s2}{\PYZdq{}}\PY{p}{]} \PY{o}{\PYZgt{}} \PY{l+m+mi}{0}\PY{p}{]}
\PY{n}{df\PYZus{}filtered}\PY{o}{.}\PY{n}{info}\PY{p}{(}\PY{p}{)}
\end{Verbatim}
\end{tcolorbox}

    \begin{Verbatim}[commandchars=\\\{\}]
<class 'pandas.core.frame.DataFrame'>
Int64Index: 15253 entries, 233 to 51190
Data columns (total 8 columns):
 \#   Column              Non-Null Count  Dtype
---  ------              --------------  -----
 0   station\_id          15253 non-null  object
 1   station\_name        15253 non-null  object
 2   date                15253 non-null  datetime64[ns]
 3   ice\_thickness       15253 non-null  float64
 4   snow\_depth          15056 non-null  float64
 5   measurement\_method  15253 non-null  float64
 6   surface\_topology    15101 non-null  float64
 7   cracks\_leads        15104 non-null  float64
dtypes: datetime64[ns](1), float64(5), object(2)
memory usage: 1.0+ MB
    \end{Verbatim}

    \begin{tcolorbox}[breakable, size=fbox, boxrule=1pt, pad at break*=1mm,colback=cellbackground, colframe=cellborder]
\prompt{In}{incolor}{15}{\boxspacing}
\begin{Verbatim}[commandchars=\\\{\}]
\PY{c+c1}{\PYZsh{} Number of records per date}
\PY{n}{alt}\PY{o}{.}\PY{n}{data\PYZus{}transformers}\PY{o}{.}\PY{n}{disable\PYZus{}max\PYZus{}rows}\PY{p}{(}\PY{p}{)}

\PY{n}{date\PYZus{}chart} \PY{o}{=} \PY{n}{alt}\PY{o}{.}\PY{n}{Chart}\PY{p}{(}\PY{n}{df\PYZus{}filtered}\PY{p}{)}\PY{o}{.}\PY{n}{mark\PYZus{}bar}\PY{p}{(}\PY{p}{)}\PY{o}{.}\PY{n}{encode}\PY{p}{(}
    \PY{n}{x} \PY{o}{=} \PY{n}{alt}\PY{o}{.}\PY{n}{X}\PY{p}{(}\PY{l+s+s2}{\PYZdq{}}\PY{l+s+s2}{date}\PY{l+s+s2}{\PYZdq{}}\PY{p}{,} \PY{n}{title}\PY{o}{=}\PY{l+s+s2}{\PYZdq{}}\PY{l+s+s2}{Day}\PY{l+s+s2}{\PYZdq{}}\PY{p}{)}\PY{p}{,}
    \PY{n}{y} \PY{o}{=} \PY{n}{alt}\PY{o}{.}\PY{n}{Y}\PY{p}{(}\PY{l+s+s2}{\PYZdq{}}\PY{l+s+s2}{count()}\PY{l+s+s2}{\PYZdq{}}\PY{p}{,} \PY{n}{title}\PY{o}{=}\PY{l+s+s2}{\PYZdq{}}\PY{l+s+s2}{Number of Measurements per Day}\PY{l+s+s2}{\PYZdq{}}\PY{p}{)}\PY{p}{,} 
    \PY{n}{tooltip} \PY{o}{=} \PY{p}{[}\PY{l+s+s2}{\PYZdq{}}\PY{l+s+s2}{date}\PY{l+s+s2}{\PYZdq{}}\PY{p}{,} \PY{l+s+s2}{\PYZdq{}}\PY{l+s+s2}{count()}\PY{l+s+s2}{\PYZdq{}}\PY{p}{]}
\PY{p}{)}\PY{o}{.}\PY{n}{properties}\PY{p}{(}
    \PY{n}{width}\PY{o}{=}\PY{l+m+mi}{1000}
\PY{p}{)}

\PY{p}{(}\PY{n}{date\PYZus{}chart} \PY{o}{\PYZam{}} \PY{n}{date\PYZus{}chart}\PY{o}{.}\PY{n}{encode}\PY{p}{(}
    \PY{n}{x} \PY{o}{=} \PY{n}{alt}\PY{o}{.}\PY{n}{X}\PY{p}{(}\PY{l+s+s2}{\PYZdq{}}\PY{l+s+s2}{yearmonth(date)}\PY{l+s+s2}{\PYZdq{}}\PY{p}{,} \PY{n}{title}\PY{o}{=}\PY{l+s+s2}{\PYZdq{}}\PY{l+s+s2}{Month}\PY{l+s+s2}{\PYZdq{}}\PY{p}{)}\PY{p}{,} 
    \PY{n}{y} \PY{o}{=} \PY{n}{alt}\PY{o}{.}\PY{n}{Y}\PY{p}{(}\PY{l+s+s2}{\PYZdq{}}\PY{l+s+s2}{count()}\PY{l+s+s2}{\PYZdq{}}\PY{p}{,} \PY{n}{title}\PY{o}{=}\PY{l+s+s2}{\PYZdq{}}\PY{l+s+s2}{Number of Measurements per Month}\PY{l+s+s2}{\PYZdq{}}\PY{p}{)}\PY{p}{,} 
    \PY{n}{tooltip} \PY{o}{=} \PY{p}{[}\PY{l+s+s2}{\PYZdq{}}\PY{l+s+s2}{yearmonth(date)}\PY{l+s+s2}{\PYZdq{}}\PY{p}{,} \PY{l+s+s2}{\PYZdq{}}\PY{l+s+s2}{count()}\PY{l+s+s2}{\PYZdq{}}\PY{p}{]}
\PY{p}{)}\PY{p}{)}\PY{o}{.}\PY{n}{properties}\PY{p}{(}
    \PY{n}{title}\PY{o}{=}\PY{l+s+s2}{\PYZdq{}}\PY{l+s+s2}{Number of Ice Thickness Measurements by Date}\PY{l+s+s2}{\PYZdq{}}\PY{p}{,}
\PY{p}{)}
\end{Verbatim}
\end{tcolorbox}

            \begin{tcolorbox}[breakable, size=fbox, boxrule=.5pt, pad at break*=1mm, opacityfill=0]
\prompt{Out}{outcolor}{15}{\boxspacing}
\begin{Verbatim}[commandchars=\\\{\}]
alt.VConcatChart({\ldots})
\end{Verbatim}
\end{tcolorbox}
        
    \textbf{Stations}

Stations vary over the years but seem relatively consistent. Some
stations seem to be replaced over time, but the stations with the
majority of measurements have records for each year.

    \begin{tcolorbox}[breakable, size=fbox, boxrule=1pt, pad at break*=1mm,colback=cellbackground, colframe=cellborder]
\prompt{In}{incolor}{16}{\boxspacing}
\begin{Verbatim}[commandchars=\\\{\}]
\PY{c+c1}{\PYZsh{} Number of stations per year}
\PY{n}{jan\PYZus{}march\PYZus{}df} \PY{o}{=} \PY{n}{df\PYZus{}filtered}\PY{o}{.}\PY{n}{query}\PY{p}{(}\PY{l+s+s2}{\PYZdq{}}\PY{l+s+s2}{date.dt.month \PYZgt{}= 1 \PYZam{} date.dt.month \PYZlt{}= 3}\PY{l+s+s2}{\PYZdq{}}\PY{p}{)}

\PY{n}{alt}\PY{o}{.}\PY{n}{Chart}\PY{p}{(}\PY{n}{jan\PYZus{}march\PYZus{}df}\PY{p}{,} \PY{n}{title} \PY{o}{=} \PY{l+s+s2}{\PYZdq{}}\PY{l+s+s2}{Stations by Year (January \PYZhy{} March only)}\PY{l+s+s2}{\PYZdq{}}\PY{p}{)}\PY{o}{.}\PY{n}{mark\PYZus{}bar}\PY{p}{(}\PY{p}{)}\PY{o}{.}\PY{n}{encode}\PY{p}{(}
    \PY{n}{x} \PY{o}{=} \PY{n}{alt}\PY{o}{.}\PY{n}{X}\PY{p}{(}\PY{l+s+s2}{\PYZdq{}}\PY{l+s+s2}{station\PYZus{}name}\PY{l+s+s2}{\PYZdq{}}\PY{p}{,} \PY{n}{title}\PY{o}{=}\PY{l+s+s2}{\PYZdq{}}\PY{l+s+s2}{Station}\PY{l+s+s2}{\PYZdq{}}\PY{p}{,} \PY{n}{sort} \PY{o}{=} \PY{l+s+s1}{\PYZsq{}}\PY{l+s+s1}{\PYZhy{}y}\PY{l+s+s1}{\PYZsq{}}\PY{p}{)}\PY{p}{,}
    \PY{n}{y} \PY{o}{=} \PY{n}{alt}\PY{o}{.}\PY{n}{Y}\PY{p}{(}\PY{l+s+s2}{\PYZdq{}}\PY{l+s+s2}{count()}\PY{l+s+s2}{\PYZdq{}}\PY{p}{,} \PY{n}{title}\PY{o}{=}\PY{l+s+s2}{\PYZdq{}}\PY{l+s+s2}{Number of Stations}\PY{l+s+s2}{\PYZdq{}}\PY{p}{)}\PY{p}{,}
    \PY{n}{color} \PY{o}{=} \PY{l+s+s2}{\PYZdq{}}\PY{l+s+s2}{year(date)}\PY{l+s+s2}{\PYZdq{}}\PY{p}{,}
    \PY{n}{tooltip} \PY{o}{=} \PY{p}{[}\PY{l+s+s2}{\PYZdq{}}\PY{l+s+s2}{year(date)}\PY{l+s+s2}{\PYZdq{}}\PY{p}{,} \PY{l+s+s2}{\PYZdq{}}\PY{l+s+s2}{count()}\PY{l+s+s2}{\PYZdq{}}\PY{p}{]}
\PY{p}{)}\PY{o}{.}\PY{n}{properties}\PY{p}{(}
    \PY{n}{width} \PY{o}{=} \PY{l+m+mi}{1000}
\PY{p}{)}
\end{Verbatim}
\end{tcolorbox}

            \begin{tcolorbox}[breakable, size=fbox, boxrule=.5pt, pad at break*=1mm, opacityfill=0]
\prompt{Out}{outcolor}{16}{\boxspacing}
\begin{Verbatim}[commandchars=\\\{\}]
alt.Chart({\ldots})
\end{Verbatim}
\end{tcolorbox}
        
    \begin{tcolorbox}[breakable, size=fbox, boxrule=1pt, pad at break*=1mm,colback=cellbackground, colframe=cellborder]
\prompt{In}{incolor}{17}{\boxspacing}
\begin{Verbatim}[commandchars=\\\{\}]
\PY{c+c1}{\PYZsh{} Distribution of ice thickness over all time}

\PY{n}{ice\PYZus{}histogram} \PY{o}{=} \PY{n}{alt}\PY{o}{.}\PY{n}{Chart}\PY{p}{(}\PY{n}{df\PYZus{}filtered}\PY{p}{,} \PY{n}{title}\PY{o}{=}\PY{l+s+s2}{\PYZdq{}}\PY{l+s+s2}{\PYZdq{}}\PY{p}{)}\PY{o}{.}\PY{n}{mark\PYZus{}bar}\PY{p}{(}\PY{p}{)}\PY{o}{.}\PY{n}{encode}\PY{p}{(}
    \PY{n}{x} \PY{o}{=} \PY{n}{alt}\PY{o}{.}\PY{n}{X}\PY{p}{(}\PY{l+s+s2}{\PYZdq{}}\PY{l+s+s2}{ice\PYZus{}thickness}\PY{l+s+s2}{\PYZdq{}}\PY{p}{,} \PY{n}{title}\PY{o}{=}\PY{l+s+s2}{\PYZdq{}}\PY{l+s+s2}{\PYZdq{}}\PY{p}{,} \PY{n+nb}{bin}\PY{o}{=}\PY{n}{alt}\PY{o}{.}\PY{n}{Bin}\PY{p}{(}\PY{n}{maxbins}\PY{o}{=}\PY{l+m+mi}{40}\PY{p}{)}\PY{p}{)}\PY{p}{,}
    \PY{n}{y} \PY{o}{=} \PY{n}{alt}\PY{o}{.}\PY{n}{Y}\PY{p}{(}\PY{l+s+s2}{\PYZdq{}}\PY{l+s+s2}{count()}\PY{l+s+s2}{\PYZdq{}}\PY{p}{,} \PY{n}{title}\PY{o}{=}\PY{l+s+s2}{\PYZdq{}}\PY{l+s+s2}{\PYZdq{}}\PY{p}{)}\PY{p}{,} 
    
\PY{p}{)}\PY{o}{.}\PY{n}{properties}\PY{p}{(}
    \PY{n}{width}\PY{o}{=}\PY{l+m+mi}{1000}
\PY{p}{)}

\PY{n}{ice\PYZus{}histogram}
\end{Verbatim}
\end{tcolorbox}

            \begin{tcolorbox}[breakable, size=fbox, boxrule=.5pt, pad at break*=1mm, opacityfill=0]
\prompt{Out}{outcolor}{17}{\boxspacing}
\begin{Verbatim}[commandchars=\\\{\}]
alt.Chart({\ldots})
\end{Verbatim}
\end{tcolorbox}
        
    \begin{tcolorbox}[breakable, size=fbox, boxrule=1pt, pad at break*=1mm,colback=cellbackground, colframe=cellborder]
\prompt{In}{incolor}{18}{\boxspacing}
\begin{Verbatim}[commandchars=\\\{\}]
\PY{c+c1}{\PYZsh{} Distribution of ice thickness for each date of interest}

\PY{n}{ice\PYZus{}histogram}\PY{o}{.}\PY{n}{properties}\PY{p}{(}
    \PY{n}{width}\PY{o}{=}\PY{l+m+mi}{200}
\PY{p}{)}\PY{o}{.}\PY{n}{facet}\PY{p}{(}
    \PY{l+s+s2}{\PYZdq{}}\PY{l+s+s2}{year(date)}\PY{l+s+s2}{\PYZdq{}}\PY{p}{,} 
    \PY{n}{columns} \PY{o}{=} \PY{l+m+mi}{5}
\PY{p}{)}
\end{Verbatim}
\end{tcolorbox}

            \begin{tcolorbox}[breakable, size=fbox, boxrule=.5pt, pad at break*=1mm, opacityfill=0]
\prompt{Out}{outcolor}{18}{\boxspacing}
\begin{Verbatim}[commandchars=\\\{\}]
alt.FacetChart({\ldots})
\end{Verbatim}
\end{tcolorbox}
        
    \begin{tcolorbox}[breakable, size=fbox, boxrule=1pt, pad at break*=1mm,colback=cellbackground, colframe=cellborder]
\prompt{In}{incolor}{19}{\boxspacing}
\begin{Verbatim}[commandchars=\\\{\}]
\PY{c+c1}{\PYZsh{} General change in ice thickness over time}

\PY{n}{ice\PYZus{}chart} \PY{o}{=} \PY{n}{alt}\PY{o}{.}\PY{n}{Chart}\PY{p}{(}\PY{n}{df\PYZus{}filtered}\PY{p}{,} \PY{n}{title}\PY{o}{=}\PY{l+s+s2}{\PYZdq{}}\PY{l+s+s2}{Mean Ice Thickness by Date}\PY{l+s+s2}{\PYZdq{}}\PY{p}{)}\PY{o}{.}\PY{n}{mark\PYZus{}line}\PY{p}{(}\PY{p}{)}\PY{o}{.}\PY{n}{encode}\PY{p}{(}
    \PY{n}{x} \PY{o}{=} \PY{n}{alt}\PY{o}{.}\PY{n}{X}\PY{p}{(}\PY{l+s+s2}{\PYZdq{}}\PY{l+s+s2}{yearmonth(date)}\PY{l+s+s2}{\PYZdq{}}\PY{p}{,} \PY{n}{title}\PY{o}{=}\PY{l+s+s2}{\PYZdq{}}\PY{l+s+s2}{Month}\PY{l+s+s2}{\PYZdq{}}\PY{p}{)}\PY{p}{,}
    \PY{n}{y} \PY{o}{=} \PY{n}{alt}\PY{o}{.}\PY{n}{Y}\PY{p}{(}\PY{l+s+s2}{\PYZdq{}}\PY{l+s+s2}{mean(ice\PYZus{}thickness)}\PY{l+s+s2}{\PYZdq{}}\PY{p}{,} \PY{n}{title}\PY{o}{=}\PY{l+s+s2}{\PYZdq{}}\PY{l+s+s2}{Mean Ice Thickness per Month}\PY{l+s+s2}{\PYZdq{}}\PY{p}{)}\PY{p}{,} 
    \PY{n}{tooltip} \PY{o}{=} \PY{p}{[}\PY{l+s+s2}{\PYZdq{}}\PY{l+s+s2}{yearmonth(date)}\PY{l+s+s2}{\PYZdq{}}\PY{p}{,} \PY{l+s+s2}{\PYZdq{}}\PY{l+s+s2}{mean(ice\PYZus{}thickness)}\PY{l+s+s2}{\PYZdq{}}\PY{p}{]}
\PY{p}{)}\PY{o}{.}\PY{n}{properties}\PY{p}{(}
    \PY{n}{width}\PY{o}{=}\PY{l+m+mi}{1000}
\PY{p}{)}

\PY{n}{ice\PYZus{}chart} \PY{o}{+} \PY{n}{ice\PYZus{}chart}\PY{o}{.}\PY{n}{mark\PYZus{}circle}\PY{p}{(}\PY{p}{)}
\end{Verbatim}
\end{tcolorbox}

            \begin{tcolorbox}[breakable, size=fbox, boxrule=.5pt, pad at break*=1mm, opacityfill=0]
\prompt{Out}{outcolor}{19}{\boxspacing}
\begin{Verbatim}[commandchars=\\\{\}]
alt.LayerChart({\ldots})
\end{Verbatim}
\end{tcolorbox}
        
    \begin{tcolorbox}[breakable, size=fbox, boxrule=1pt, pad at break*=1mm,colback=cellbackground, colframe=cellborder]
\prompt{In}{incolor}{59}{\boxspacing}
\begin{Verbatim}[commandchars=\\\{\}]
\PY{n}{alt}\PY{o}{.}\PY{n}{Chart}\PY{p}{(}\PY{n}{df\PYZus{}filtered}\PY{p}{,} \PY{n}{title}\PY{o}{=}\PY{l+s+s2}{\PYZdq{}}\PY{l+s+s2}{Mean Ice Thickness by Date}\PY{l+s+s2}{\PYZdq{}}\PY{p}{)}\PY{o}{.}\PY{n}{mark\PYZus{}boxplot}\PY{p}{(}\PY{p}{)}\PY{o}{.}\PY{n}{encode}\PY{p}{(}
    \PY{n}{x} \PY{o}{=} \PY{n}{alt}\PY{o}{.}\PY{n}{X}\PY{p}{(}\PY{l+s+s2}{\PYZdq{}}\PY{l+s+s2}{ice\PYZus{}thickness}\PY{l+s+s2}{\PYZdq{}}\PY{p}{,} \PY{n}{title}\PY{o}{=}\PY{l+s+s2}{\PYZdq{}}\PY{l+s+s2}{Ice Thickness}\PY{l+s+s2}{\PYZdq{}}\PY{p}{)}\PY{p}{,}
    \PY{n}{y} \PY{o}{=} \PY{n}{alt}\PY{o}{.}\PY{n}{Y}\PY{p}{(}\PY{l+s+s2}{\PYZdq{}}\PY{l+s+s2}{station\PYZus{}name}\PY{l+s+s2}{\PYZdq{}}\PY{p}{,} \PY{n}{title}\PY{o}{=}\PY{l+s+s2}{\PYZdq{}}\PY{l+s+s2}{Station}\PY{l+s+s2}{\PYZdq{}}\PY{p}{,} \PY{n}{sort}\PY{o}{=}\PY{l+s+s2}{\PYZdq{}}\PY{l+s+s2}{\PYZhy{}x}\PY{l+s+s2}{\PYZdq{}}\PY{p}{)}\PY{p}{,}
\PY{p}{)}\PY{o}{.}\PY{n}{properties}\PY{p}{(}
    \PY{n}{height}\PY{o}{=}\PY{l+m+mi}{1600}
\PY{p}{)}
\end{Verbatim}
\end{tcolorbox}

            \begin{tcolorbox}[breakable, size=fbox, boxrule=.5pt, pad at break*=1mm, opacityfill=0]
\prompt{Out}{outcolor}{59}{\boxspacing}
\begin{Verbatim}[commandchars=\\\{\}]
alt.Chart({\ldots})
\end{Verbatim}
\end{tcolorbox}
        
    \begin{tcolorbox}[breakable, size=fbox, boxrule=1pt, pad at break*=1mm,colback=cellbackground, colframe=cellborder]
\prompt{In}{incolor}{62}{\boxspacing}
\begin{Verbatim}[commandchars=\\\{\}]
\PY{n}{alt}\PY{o}{.}\PY{n}{Chart}\PY{p}{(}\PY{n}{df\PYZus{}filtered}\PY{p}{,} \PY{n}{title}\PY{o}{=}\PY{l+s+s2}{\PYZdq{}}\PY{l+s+s2}{Ice Thickness \PYZhy{} All Dates}\PY{l+s+s2}{\PYZdq{}}\PY{p}{)}\PY{o}{.}\PY{n}{mark\PYZus{}boxplot}\PY{p}{(}\PY{p}{)}\PY{o}{.}\PY{n}{encode}\PY{p}{(}
    \PY{n}{x}\PY{o}{=}\PY{n}{alt}\PY{o}{.}\PY{n}{X}\PY{p}{(}\PY{l+s+s2}{\PYZdq{}}\PY{l+s+s2}{ice\PYZus{}thickness}\PY{l+s+s2}{\PYZdq{}}\PY{p}{,} \PY{n}{title}\PY{o}{=}\PY{l+s+s2}{\PYZdq{}}\PY{l+s+s2}{Ice Thickness}\PY{l+s+s2}{\PYZdq{}}\PY{p}{)}\PY{p}{,}
    \PY{n}{tooltip}\PY{o}{=}\PY{l+s+s2}{\PYZdq{}}\PY{l+s+s2}{date}\PY{l+s+s2}{\PYZdq{}}
\PY{p}{)}\PY{o}{.}\PY{n}{properties}\PY{p}{(}
    \PY{n}{height}\PY{o}{=}\PY{l+m+mi}{300}
\PY{p}{)}
\end{Verbatim}
\end{tcolorbox}

            \begin{tcolorbox}[breakable, size=fbox, boxrule=.5pt, pad at break*=1mm, opacityfill=0]
\prompt{Out}{outcolor}{62}{\boxspacing}
\begin{Verbatim}[commandchars=\\\{\}]
alt.Chart({\ldots})
\end{Verbatim}
\end{tcolorbox}
        
    \begin{tcolorbox}[breakable, size=fbox, boxrule=1pt, pad at break*=1mm,colback=cellbackground, colframe=cellborder]
\prompt{In}{incolor}{ }{\boxspacing}
\begin{Verbatim}[commandchars=\\\{\}]

\end{Verbatim}
\end{tcolorbox}


    % Add a bibliography block to the postdoc
    
    
    
\end{document}
